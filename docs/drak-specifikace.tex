\documentclass{article}
\usepackage[utf8]{inputenc}
\usepackage[czech]{babel}
\usepackage{graphicx}

\title{TODONAZEV \\ Specifikace a plán implementace ročníkového projektu}
\author{Kateřina Nevolová \\ \texttt{katka.nevolova@gmail.com} \\ 73502469}

\begin{document}
\maketitle

\section{Základní informace}
Hra TODONAZEV spočívá v ničení a vyhlazování různých produktů lidské civilizace pomocí draka. Hráč může používat běžné dračí schopnosti (oheň, létání) a získávat skóre za destrukci věrné simulace lidského města.

Tento projekt si klade za cíl návrh a implementaci hry, finálním výsledkem bude tradiční herní balíček pro MS Windows i Linux podporující hru jednoho hráče ve volných levelech a hru v kampani.

\subsection{Použité technologie}

\begin{itemize}
\item OpenGL knihovny a GLUT na vykreslování grafiky
\item OpenAL pro produkci zvuku
\item C++
\item knihovny pro manipulaci s obrázky pro načítání textur
\end{itemize}

\section{Popis softwarového díla}

\subsection{Motivace}
Důvodem vzniku byla oblíbenost jednoduchých a fyzikálně realisticky
vypadajících her, které hráčům umožňují něco ničit -- příkladem budiž známí
Angry Birds nebo Fruit Ninja. Oblíbený druh ničivé zábavy bohužel ještě nikde
nebyl dotažen k úplné dokonalosti, hra TODONAZEV je myšlena jako pokus se
ideálu přiblížit z další strany zpracováním jiného prostředí a jiných myšlenek
a herních principů.

\subsection{Uživatelské rozhraní}

Protože hra je míněná především pro hráčskou veřejnost, bude kladen důraz na zachování ``tradičního'' a všem známého přístupu k ovládání. Program nabídne hlavní menu s různými volbami (viz. níže), po začátku samotné hry se drak bude ovládat myší a klávesnicí.

\subsubsection{Hlavní menu}

kampan/volna hra/generovane levely

\subsubsection{Herní rozhraní}

ridis draka mysi + space
esc > menu
strely - 2 typy - levy cudl - plamenomet X pravy - chvili drzet - load fireboltu
(+ nejaka extra magie podle pohybu draka pri loadu fireboltu)
tab na help k nejblizsi veci co se jeste musi znicit

ovladani/prehled o deni

\subsection{Herní prostředí}

snih, dest (ovlivneni delku horeni) 

\subsubsection{Drak}

ovladani - jako letadylko,+ mezernik na boost

\subsubsection{Oheň}

strely - plamen - cara/koule

\subsubsection{Detailní popis mapy}

znicitelne veci na mape bude jeden typ objektu, trida na vec co de znicit (barak, strom, clovek,..) X teren (heightmap)

\subsubsection{Nepřátelé}

domy, strileci domy, lidi, strileci lidi, stromy, lesy

\section{Reference}

\section{Popis implementace}

fonty knihovna oglft, 
opengl + glut
vstup glut

format mapoveho souboru: 
(okraj mapy jako v hillsbradu, mlha houstne, pak neprostupno)
TODO
obrazek heightmapy + obrazek terenu + souradnice nepratel

textury v png (steny baraku, strechy, stromy, drak, ale lidi asi uboze)
modely vseho proceduralne
prostredi a teren taky proceduralne

navigace jako chcipaci tunel X maso ..

zabavna statistika po misi (zabiti, zapaleni ..)

vsechny zobrazovany objekty - jedna trida odvozeny
vsechny objekty pripojovat na quadtree

inteligence lidi a kolize s okolim (domy)
vic typu lidi (clovek se zbrani, clovek s vlastni atomovkou)
atomic bomber raketa

fyzikalni model draka odzkouset zvlast nejdrif

particle system

kamera!

\end{document}

